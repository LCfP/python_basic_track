% Activate the following line by filling in the right side. If for example the name of the root file is Main.tex, write
% "...root = Main.tex" if the chapter file is in the same directory, and "...root = ../Main.tex" if the chapter is in a subdirectory.
 
%!TEX root =  ../../python_basic_track.tex

\chapter{Strings \& Console Output}

	\section*{TL;DR}
	
	\begin{itemize}
		\item We can combine simple variables, like characters;
		\item into more complex ones, like strings.
		\item 
	\end{itemize}
	\clearpage

	\section{Strings}
	
	In the previous chapter you have seen some basic data types.
	One of the basic types is the `character'.
	As the name suggests this type of variable can represent any\footnote{Clearly there is a limited `characterset', but if you stay within the characters used in English you should be safe. More about this topic later.} single character.
	Often being able to represent a single character will not be enough.
	After all, characters are usually combined to form words, sentences, paragraphs, etc.
	
	To help us do this a new type of variabel was created. 
	It is called a \textbf{\index{string}} because it represents a string of characters.
	This concept is available in most (if not all) programming language but can have slight variations. 
	Here we will focus on how strings work in python.
	
	To create a string we need to tell the system where the string starts and were it ends. 
	Like in most languages you can use \lstinline|"| and \lstinline|'|.
	Since these are characters themselves we cannot just use them inside of a string.
	We need to `escape' them by putting a \lstinline|\| in front of them.
	That makes \lstinline|\| a special character in its own right requiring it to be escaped as well.
	An overview of common escape sequences is given in \autoref{table:escape}.
	
	\begin{table}[htb]
		\centering
		\caption{Common escape sequences}\label{table:escape}
		\begin{tabular}{ l | l }
			Sequence	& Represents 	\\ \hline
			\lstinline|"\'"|	& \verb|'|	\\ \hline
			\lstinline|"\""|	& \verb|"|	\\ \hline
			\lstinline|"\\"|	& \verb|\|	\\ \hline
			\lstinline|"\n"|	& newline	\\ \hline
			\lstinline|"\t"|	& tab	\\
			%\lstinline|"\N{name}"|	& tab	\\ \hline % UNICODE ONLY!!! introducing this will likely lead mostly to confusion not to learning...
		\end{tabular}
	\end{table}
	
	\section{Index}
	
	% in computers counting typically happens as an offset so if we want the 2nd item we say firstItemplus[1], the first item is firstItemplus[0]
	
	\section{(String) Methods}
	
	%len() lower() upper() str()
	
	\section{Print}
	
		\subsection{Concatenation}
		
		\subsection{Explicit string conversion}
		
		\subsection{String formatting}
		
	\section{Apply these concepts}
	
		\subsection{Date and Time}
		
		% actual mm/dd/yyyy '%02d/%02d/%04d'